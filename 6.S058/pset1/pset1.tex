% \documentclass{article}
\documentclass[a4paper]{article}
\usepackage{/Users/mariachrysafis/Documents/Schoolwork/18.650/chrysafis}
\graphicspath{ {/} }
\author{Maria Chrysafis}
\date{\today}
\title{Computer Vision 6.S058 Problem Set 1}
\begin{document}
\maketitle
\section{Problem 1}
\section{Problem 2}
We are in effect given the equation
\begin{align*}
\begin{bmatrix}
    x \\ y
\end{bmatrix}
= \alpha \cdot P \cdot R_x (\theta) \cdot \begin{bmatrix}
    X \\ Y \\ Z
\end{bmatrix}
+ \begin{bmatrix}
    x_0 \\ y_0
\end{bmatrix},
\end{align*}
where $R_x(\theta)$ represents a rotation amtrix around the $X$-axis by an angle $\theta$, $P$ is a projection matrix that reduces 3D world coordinates to 2D image coordinates, $\alpha$ is a scaling factor to account for the camera sensor size, and $(x_0, y_0)$ represents the image coordinates of the origin of the camera coordinate system.

In this case, the rotation matrix around the $X$ axis by an angle $\theta$ is 
\begin{align*}
    R_x(\theta) = \begin{bmatrix}
    1 & 0 & 0 \\
    0 & \cos \theta & -\sin \theta \\
    0 & \sin \theta & \cos \theta
    \end{bmatrix}
\end{align*}
and the projection matrix is
\begin{align*}
    P = \begin{bmatrix}
        1 & 0 & 0 \\
        0 & \cos \theta & -\sin \theta \\
    \end{bmatrix}.
\end{align*}
Therefore, this equation becomes
\begin{align*}
    \begin{bmatrix}
        x \\ y
    \end{bmatrix}
    &= \alpha \cdot \begin{bmatrix}
        1 & 0 & 0 \\
        0 & \cos \theta & -\sin \theta \\
    \end{bmatrix} \cdot \begin{bmatrix}
        1 & 0 & 0 \\
        0 & \cos \theta & -\sin \theta \\
        0 & \sin \theta & \cos \theta
        \end{bmatrix} \cdot \begin{bmatrix}
        X \\ Y \\ Z
    \end{bmatrix}
    + \begin{bmatrix}
        x_0 \\ y_0
    \end{bmatrix}
    \\ &= \alpha \cdot
    \begin{bmatrix}
        1 & 0 & 0 \\
        0 & \cos^2 \theta & -\sin^2 \theta & -2 \cos \theta \sin \theta \\
    \end{bmatrix}\cdot \begin{bmatrix}
        X \\ Y \\ Z
    \end{bmatrix}+ \begin{bmatrix}
        x_0 \\ y_0
    \end{bmatrix}
    \\ &= \begin{bmatrix}
        \alpha (X + x_0) \\
        \alpha (\cos \theta Y - \sin \theta Z + y_0)
    \end{bmatrix}.
\end{align*}
We know that $(0, 0, 0)$ projects to $(0, 0)$, i.e. when $X = 0, Y = 0$, and $z = 0$, we get $x = 0$ and $y = 0$. Therefore, the equation becomes
\begin{align*}
    x &= \alpha (0) + x_0 = 0 \Longrightarrow x_0 = 0  \\
    y &= \alpha (\cos \theta (0)-\sin \theta (0)) + y_0 = 0 \Longrightarrow y_0 = 0.
\end{align*}
So, $x_0 = 0$ and $y_0 = 0$. Now, applying the second condition (that $(1, 0, 0)$ projects to $(3, 0)$), we get that
\begin{align*}
    x = \alpha(1) + x_0 = 3 \Longrightarrow \alpha = 3 \\
    y = \alpha(\cos \theta (1)-\sin \theta (0)) + y_0.
\end{align*}
So, $\alpha = 3$, $x_0 = 0$, and $y_0 = 0$.
\end{document}  