% \documentclass{article}
\documentclass[a4paper]{article}
\usepackage{/Users/mariachrysafis/Documents/Schoolwork/18.650/chrysafis}
\graphicspath{ {/} }
\author{Maria Chrysafis}
\date{\today}
\title{Computer Vision 6.S058 Problem Set 1}
\begin{document}
\maketitle
\section{Problem 1}
\section{Problem 2}
We are in effect given the equation
\begin{align*}
\begin{bmatrix}
    x \\ y
\end{bmatrix}
= \alpha \cdot P \cdot R_x (\theta) \cdot \begin{bmatrix}
    X \\ Y \\ Z
\end{bmatrix}
+ \begin{bmatrix}
    x_0 \\ y_0
\end{bmatrix},
\end{align*}
where $R_x(\theta)$ represents a rotation amtrix around the $X$-axis by an angle $\theta$, $P$ is a projection matrix that reduces 3D world coordinates to 2D image coordinates, $\alpha$ is a scaling factor to account for the camera sensor size, and $(x_0, y_0)$ represents the image coordinates of the origin of the camera coordinate system.

In this case, the rotation matrix around the $X$ axis by an angle $\theta$ is 
\begin{align*}
    R_x(\theta) = \begin{bmatrix}
    1 & 0 & 0 \\
    0 & \cos \theta & -\sin \theta \\
    0 & \sin \theta & \cos \theta
    \end{bmatrix}
\end{align*}
and the projection matrix is
\begin{align*}
    P = \begin{bmatrix}
        1 & 0 & 0 \\
        0 & \cos \theta & -\sin \theta \\
    \end{bmatrix}.
\end{align*}
Therefore, this equation becomes
\begin{align*}
    \begin{bmatrix}
        x \\ y
    \end{bmatrix}
    &= \alpha \cdot \begin{bmatrix}
        1 & 0 & 0 \\
        0 & \cos \theta & -\sin \theta \\
    \end{bmatrix} \cdot \begin{bmatrix}
        1 & 0 & 0 \\
        0 & \cos \theta & -\sin \theta \\
        0 & \sin \theta & \cos \theta
        \end{bmatrix} \cdot \begin{bmatrix}
        X \\ Y \\ Z
    \end{bmatrix}
    + \begin{bmatrix}
        x_0 \\ y_0
    \end{bmatrix}
    \\ &= \alpha \cdot
    \begin{bmatrix}
        1 & 0 & 0 \\
        0 & \cos^2 \theta & -\sin^2 \theta & -2 \cos \theta \sin \theta \\
    \end{bmatrix}\cdot \begin{bmatrix}
        X \\ Y \\ Z
    \end{bmatrix}+ \begin{bmatrix}
        x_0 \\ y_0
    \end{bmatrix}
    \\ &= \begin{bmatrix}
        \alpha (X + x_0) \\
        \alpha (\cos \theta Y - \sin \theta Z + y_0)
    \end{bmatrix}.
\end{align*}
We know that $(0, 0, 0)$ projects to $(0, 0)$, i.e. when $X = 0, Y = 0$, and $z = 0$, we get $x = 0$ and $y = 0$. Therefore, the equation becomes
\begin{align*}
    x &= \alpha (0) + x_0 = 0 \Longrightarrow x_0 = 0  \\
    y &= \alpha (\cos \theta (0)-\sin \theta (0)) + y_0 = 0 \Longrightarrow y_0 = 0.
\end{align*}
So, $x_0 = 0$ and $y_0 = 0$. Now, applying the second condition (that $(1, 0, 0)$ projects to $(3, 0)$), we get that
\begin{align*}
    x = \alpha(1) + x_0 = 3 \Longrightarrow \alpha = 3 \\
    y = \alpha(\cos \theta (1)-\sin \theta (0)) + y_0.
\end{align*}
So, $\alpha = 3$, $x_0 = 0$, and $y_0 = 0$.
\section{Problem 3}
We know that
$$Z(x, y) = \frac{Y(x, y) \cos \theta}{\sin \theta}- \frac{y}{\sin \theta}.$$
\subsection{Constraint Along Vertical Edges}
We know that 
\begin{align*}
\frac{\partial Z}{\partial y} &= \frac{\partial}{\partial y} \left( \frac{Y \cos \theta}{\sin \theta} - \frac{y}{\sin \theta} \right) 
\\ &= \frac{\cos \theta}{\sin \theta} \cdot \frac{\partial Y}{\partial y} - \frac{1}{\sin \theta}
\\ &= \frac{\cos \theta}{\sin \theta} \cdot \frac{1}{\cos \theta} - \frac{1}{\sin \theta}
\\ &= 0.
\end{align*}
\subsection{Constraint Along Horizontal Edges}
For the constraint along horizontal edges,
\begin{align*}
    \frac{\partial Z}{\partial t} &= \frac{\partial}{\partial t} \left( \frac{Y \cos \theta}{\sin \theta} - \frac{y}{\sin \theta} \right)
    \\ &= \frac{\cos \theta}{\sin \theta} \cdot \frac{\delta Y}{\delta t} - \frac{1}{\sin \theta} \cdot \frac{\delta y}{\delta t}.
\end{align*}
Since $\frac{\delta Y}{\delta t} = 0$ (from the horizontal edge constraint) and $\frac{\delta y}{\delta t} = n_x$ (because $t = (-n_y, n_x)$),
$$\frac{\partial Z}{\partial t} = 0 - \frac{n_x}{\sin \theta} = - \frac{n_x}{\sin \theta}.$$
\subsection{Constraint on Flat Surfaces}
For flat surfaces, the second derivative of $Y(x, y)$ is zero, so
\begin{align*}
    \frac{\partial^2 Y}{\partial x^2} = \frac{\partial^2 Y}{\partial y^2} = \frac{\partial^2 Y}{\partial x \partial y} = 0.
\end{align*}
Some algebra shows that
\begin{align*}
    \frac{\partial^2 Z}{\partial x^2} = \frac{\partial^2}{\partial x^2} \left( \frac{Y \cos \theta}{\sin \theta} - \frac{y}{\sin \theta} \right) = 0 \\
    \frac{\partial^2 Z}{\partial x^2} = \frac{\partial^2}{\partial y^2} \left( \frac{Y \cos \theta}{\sin \theta} - \frac{y}{\sin \theta} \right) = 0 \\
    \frac{\partial^2 Z}{\partial x^2} = \frac{\partial^2}{\partial x \partial y} \left( \frac{Y \cos \theta}{\sin \theta} - \frac{y}{\sin \theta} \right) = 0.
\end{align*}
So, 
$$\frac{\partial^2 Z}{\partial x^2} = \frac{\partial^2 Z}{\partial y^2} = \frac{\partial^2 Z}{\partial x \partial y} = 0.$$
\section{Problem 4}
\section{Problem 5}
\section{Problem 6}
\section{Problem 7}
\subsection*{Part (a): Equations Relating Angles of Rays Passing Through a Plano-Convex Lens}

For a plano-convex lens, the front surface has a radius of curvature \( R \), and the back surface is flat. The equations relating the angles of rays passing through the lens are derived using Snell's Law and the paraxial approximation (small angles).

\begin{enumerate}
    \item \textbf{Snell's Law at the First Surface (Curved Surface):}
    \[
    n_1 \sin(\theta_1) = n_2 \sin(\theta_2)
    \]
    Where:
    \begin{itemize}
        \item \( n_1 \) is the refractive index of the medium before the lens (usually air, \( n_1 \approx 1 \)).
        \item \( n_2 \) is the refractive index of the lens material.
        \item \( \theta_1 \) is the angle of incidence.
        \item \( \theta_2 \) is the angle of refraction.
    \end{itemize}

    \item \textbf{Angle of Incidence at the First Surface:}
    \[
    \theta_1 = \alpha + \phi
    \]
    Where:
    \begin{itemize}
        \item \( \alpha \) is the angle the ray makes with the optical axis before refraction.
        \item \( \phi \) is the angle the normal to the surface makes with the optical axis.
    \end{itemize}

    \item \textbf{Angle of Refraction at the First Surface:}
    \[
    \theta_2 = \phi + \beta
    \]
    Where:
    \begin{itemize}
        \item \( \beta \) is the angle the refracted ray makes with the optical axis.
    \end{itemize}

    \item \textbf{Snell's Law at the Second Surface (Flat Surface):}
    \[
    n_2 \sin(\beta) = n_1 \sin(\gamma)
    \]
    Where:
    \begin{itemize}
        \item \( \gamma \) is the angle of the ray after exiting the lens.
    \end{itemize}

    \item \textbf{Paraxial Approximation:}
    \[
    \sin(\theta) \approx \theta \quad \text{(for small angles)}
    \]
    This approximation simplifies the trigonometric relationships, making the equations linear and easier to solve.
\end{enumerate}

\subsection*{Part (b): Expression for the Lens Focal Length}

The lensmaker's formula for a thin lens is given by:
\[
\frac{1}{f} = (n - 1) \left( \frac{1}{R_1} - \frac{1}{R_2} \right)
\]
Where:
\begin{itemize}
    \item \( f \) is the focal length of the lens.
    \item \( n \) is the refractive index of the lens material.
    \item \( R_1 \) and \( R_2 \) are the radii of curvature of the two lens surfaces.
\end{itemize}

For a plano-convex lens:
\begin{itemize}
    \item \( R_1 = R \) (radius of curvature of the curved surface).
    \item \( R_2 = \infty \) (since the back surface is flat).
\end{itemize}

Substituting these into the lensmaker's formula:
\[
\frac{1}{f} = (n - 1) \left( \frac{1}{R} - \frac{1}{\infty} \right) = (n - 1) \left( \frac{1}{R} - 0 \right) = \frac{n - 1}{R}
\]
Therefore, the focal length \( f \) of the plano-convex lens is:
\[
f = \frac{R}{n - 1}
\] 
\end{document} 