\documentclass[a4paper]{article}
\usepackage{/Users/mariachrysafis/Documents/Schoolwork/6.1800/chrysafis}
\graphicspath{ {/} }

\begin{document}

\section*{Recitation 7: Ethernet}

\begin{Exercise}
    What problem are the authors trying to solve?
\end{Exercise} 
\begin{Solution}
    The authors aim to develop a simple, scalable, and cost-effective system for interconnecting multiple computers within a local network. At the time, there was a need for a reliable way to allow multiple machines to communicate without excessive wiring or centralized control, particularly in environments like research labs or offices.

    Key challenges they address:
    \begin{itemize}
    \item Efficient sharing of a communication medium (minimizing collisions and delays)
    \item Decentralized control (avoiding single points of failure)
    \item Scalability (supporting more devices without significantly increasing complexity)
    \item Broadcast and packet switching capabilities (for flexibility in networked applications)
    \end{itemize}
\end{Solution}
\begin{Exercise}
    What choices do they make, and how do they explain or justify those choices?
\end{Exercise}
\begin{Solution}
\begin{itemize}
    \item Carrier Sense Multiple Access with Collision Detection (CSMA/CD)
    \begin{itemize}
        \item Ethernet uses CSMA/CD, meaning:
        \begin{itemize}
            \item Devices listen before transmitting (Carrier Sense).
            \item Multiple devices can access the medium (Multiple Access).
            \item If two devices transmit simultaneously, they detect the collision and retry (Collision Detection).
            \item Justification: Avoids the need for central coordination and ensures fairness.
        \end{itemize}
    \end{itemize}
    \item Packet-Switched Rather Than Circuit-Switched Design
    \begin{itemize}
        \item Data is sent in discrete packets rather than maintaining a dedicated connection.
        \item Justification: More efficient use of network bandwidth and supports dynamic communication patterns. 
    \end{itemize}
    \item Exponential Backoff for Collision Handling
    \begin{itemize}
        \item After a collision, each sender waits a random amount of time before retrying, with the waiting time increasing exponentially if collisions persist.
        \item Justification: Prevents repeated collisions and helps the network recover from congestion.
    \end{itemize}
    \item Use of a Shared Coaxial Cable as a Broadcast Medium
    \begin{itemize}
        \item All devices connect to a single coaxial cable where packets are broadcasted and addressed.
        \item Justification: Simplicity, cost-effectiveness, and ease of adding more devices.
    \end{itemize}
\end{itemize}
\end{Solution}
\begin{Exercise}
    Do they mention alternatives? If so, what do they say is undesirable about those alternatives?
\end{Exercise}
\begin{Solution}
    \begin{itemize}
        \item Time Division Multiplexing (TDM). Assigns fixed time slots for communication. Drawback: Inefficient when devices do not always need to transmit, leading to wasted slots. 
        \item Centralized Control (e.g., Token Ring). A central entity or token-passing scheme determines when a device can transmit. Drawback: Adds complexity, increases latency, and introduces potential failure points.
        \item Circuit-Switched Networks. Establishes dedicated end-to-end connections. Drawback: Inefficient for bursty data traffic and limits flexibility.
    \end{itemize}
\end{Solution}
\begin{Exercise}
    What is the connection between this paper and lecture?
\end{Exercise}
\begin{Solution}
    The paper discusses efficiency, scalability, and fairness. We dsicussed these topics in lecture 1. Relating to the textbook chapters:
    \begin{itemize}
    \item Packet Forwarding & Delay (7.1.2): Ethernet uses packet-switching rather than circuit-switching, which introduces potential delays due to collisions and retransmissions in CSMA/CD.
    \item Buffer Overflow & Discarded Packets (7.1.3): Since Ethernet does not guarantee delivery, packets may be discarded if the receiver is overwhelmed, requiring higher-layer protocols to handle retransmission.
    \item Duplicate Packets & Duplicate Suppression (7.1.4): Collisions and retransmissions can lead to duplicate packets, a problem mitigated by mechanisms at higher layers.
    \item Damaged Packets & Broken Links (7.1.5): Ethernet’s reliance on electrical signals over coaxial cable means that noise and physical damage can corrupt packets, necessitating error detection (e.g., CRC checks).
    \end{itemize}
\end{Solution}
\end{document}

